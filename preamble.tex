% Модификатор - команда стиля, действующая, пока не сказано иное (закончится
% область видимости или противоположный модификатор).

\documentclass[a4paper,cmcyralt,12pt]{report}

% КОДИРОВКА И ЛОКАЛИЗАЦИЯ
\usepackage[english,russian]{babel}	% локализация и переносы
\usepackage[utf8]{inputenc} % кодировка исходного кода (input)
\usepackage[T2A]{fontenc} % кодировка выходного текста (output)

% GEOMETRY
\usepackage{geometry}
\geometry{top = 25mm}
\geometry{bottom = 30mm}
\geometry{left = 20mm}
\geometry{right = 20mm}
% Есть и другие параметры (не из пакета geometry):
\linespread{0.7} % междустрочный интервал
\setlength{\parindent}{3ex} % отступ 1-ой строки абзаца
\setlength{\parskip}{2mm} % отступ после начала нового абзаца
% \geometry{paperheight = 297mm}

% КОЛОНТИТУЛЫ
\usepackage{titleps}
\newpagestyle{main}{
  % толщина линии
  \setheadrule{0.4pt}
  % \thepage - номер текущей страницы
  \sethead{лево}{\thepage}{право}
  \setfootrule{5pt}
  \setfoot{лево}{центр}{право}
}
% С этого момента все страницы, если не сказано иное, имеют данный стиль.
\pagestyle{main}

% ШРИФТЫ И НАЧЕРТАНИЯ
\usepackage{soulutf8}

% МАТЕМАТИКА
\usepackage{mathtext} % русские буквы в формулах
\usepackage{amsmath,amsfonts} % математические шрифты
\usepackage{amsthm} % теоремы

% ТЕОРЕМЫ
\theoremstyle{plain}
\newtheorem{theorem}{Моя теорема}
\newtheorem{lemma}{Моя лемма}

% РАЗНОЕ
\usepackage{cmap} % поиск по pdf
\usepackage{hyperref} % кликабельные ссылки
\usepackage{graphicx} % для графики
\usepackage{lipsum} % Lorem ipsum...
\usepackage{alltt} % не игнорирует пробелы
\usepackage{indentfirst} % отступ в начале параграфа

% КАСТОМНЫЕ КОМАНДЫ
% Новая команды с уникальным именем:
\newcommand{\deriv}[2]{\frac{\partial{#1}}{\partial{#2}}}
\newcommand{\R}{\mathbb{R}}
% Переопределение команд (полезно для некоторых греческих букв с непривычными
% очертаниями):
\renewcommand{\phi}{\varphi}
\renewcommand{\epsilon}{\varepsilon}
% Математические операторы (внутри формулы печатаются без курсива и ставят
% пробел перед операндом):
\DeclareMathOperator{\Kerr}{Ker} % так как \Ker может быть занято
% Со "*" создаёт оператор, верхних и нижний индексы которого будут являться
% переделами:
% WARN Не работает для внутристрочных формул!
\DeclareMathOperator*{\argmax}{argmax}
% Бинарные операторы (контролирует расстановку пробелов между операндами):
\newcommand{\rem}{\mathbin{\%}}

% КАСТОМНЫЕ ОКРУЖЕНИЯ
% Имя, что идёт перед содержимым и после:
\newenvironment{centerbf}{\begin{center}\bfseries}{\end{center}}

% ARTICLE
% \title{Название}
% \author{Имя автора}
% \date{\today}
